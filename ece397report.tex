%\documentclass{article}
%\usepackage[utf8]{inputenc}

%\title{ECE 397 Report}
%\author{ }
%\date{December 2014}

%\begin{document}

%\maketitle

%\section{Introduction}

%\end{document}
\title{ECE 397 Research Report}
\author{
        Bhargava Manja \\
        manja2@illinois.edu\\
}
\date{\today}

\documentclass[11pt]{article}

\begin{document}
\maketitle

\section{Introduction}
This report aims to summarize my work with the DEPEND research group from
October 2014 to December 2014. I first discuss the background of the project,
and in Section II I present a review of related work. An overview of concolic
execution and S2E is presented in Section III. In Section IV, I discuss the
details of QEMU and KVM and how they fit together in this project. Section V
presents plans for future work. 

\subsection{Background}
My research aims to use concolic execution to
test the KVM hypervisor. In the context of this work, \textit{concolic
execution} is a testing technique that performs symbolic execution, which
treats program variables as symbolic values, along concrete execution
paths. By combining symbolic execution, which can test whole classes of
inputs at once, with selective concrete execution, we can test a large number of
machine states and catch bugs that are simply impractical to find with unit
tests \par
 
I met with Zak and Cuong late last semester about working with DEPEND, but the
process of selecting a project did not begin until early this semester. After
much discussion, we chose to work on using concolic execution to test parts of
KVM, due to my interest in programming languages and symbolic execution. \par
This
semester, I focused on understanding S2E, the concolic execution engine we
decided to use, and how it could be used to test KVM. Due to time constraints
and the process of learning, I performed mostly exploratory work, which will be
expanded upon next semester.

\subsection{Timeline}
Details on each of the following items are presented throughout the rest of the
report
\begin{itemize}
\item Joined the group and discussed potential projects
\item Began learning about concolic execution, the architecture of KVM, and S2E,
  a concolic execution engine that was to be used for the project
\item Set up a Linux machine and completed a majority of the S2E tutorials
\item Studied the implementation of AMD SVM support in QEMU to ascertain whether
  it would be feasible to add support for the Intel VT-X instruction set
\item Created a two tiered testing setup to test KVM's hardware acceleration
  features with S2E
\item Identified code paths to begin instrumentation for S2E's concolic
  execution engine
\item TODO: ran experiment
\item 
\end{itemize}

\section{Concolic Execution}
In \textit{unit testing}, programs are decomposed into functional units which
are individually tested by generating inputs to functions that serve as entry
points. Thus, assuming a collection of unit tests cover all code paths, the
quality of testing reduces to the quality of test inputs. Manual generation of
values leads to incomplete testing and necessitates changing inputs whenever the
system changes, which is impractical. \par
Several techniques are used to automate
the production of high quality testing inputs. One scheme randomly chooses
values from the space of possible inputs. This has two serious problems: first,
many values may lead to the same states and behaviour, and second, the
probability of selecting inputs that lead to bad states and buggy behaviour is
often incredibly small, especially for any sizeable real world system that may
have millions of discrete states. \par
Another approach that eliminates redundant
test input and improves code coverage is \textit{symbolic execution}, in which a
program is executed with symbolic values as input. Each conditional expression
represents a constraint that determines an execution path. Thus, the states of
the program can be represented as an execution tree, where internal nodes
represent conditional branches. By generating concrete values, we want to
explore all the paths in the tree. However, as you can imagine, for any large
software system the size of the execution tree is intractably large, and so can
not be explored purely symbolically. A program with just ten binary branches
yields $2^{10}$ different states. \par
First proposed by Larson and Astin, (TODO
cite) \textit{concolic execution} modifies the concept of symbolic execution to
exploit its advantages while avoiding the combinatorial explosion of program
states. With this approach, concrete execution accrues constraints at every
branch point. These contraints are solved to generate new concrete inputs that
direct execution along new paths. This idea has been refined, notably by
Godefroid et. al and Sen, Marinov, and Agha here at the University of Illinois. 

\section{Literature Review}
Work has been done by other researchers pertaining to the use of concolic
execution in testing. In particular, previous related work
tends to focus on one of the following:
\begin{itemize}
\item Strategies to increase the effectiveness of concolic execution as a bug
finding tool
\item Concolic execution based testing of different classes of applications
\item Comparisons between concolic testing and other forms of testing
\item Novel uses of concolic execution analysis
\end{itemize}

\subsection{Increasing Testing Effectiveness}
Because of both the inherent ineffeciency of using and keeping track of symbolic
variables during program execution, much work has been done for the purpose of
making concolic testing more effecient. Ineffeciencies exist because of the the
sheer size of the space of possible program states. Many methods have been
developed to prune the state tree. Flanagan and Godefroid introduce the idea
of \textit{partial order reduction}, which identifies branches that do not
depend on each other and utilizes concurrency to speed exploration and execution
of these branches. The ``Boosting Concolic Testing'' paper develops the idea
of \textit{interpolation}, which 

\section{S2E}
S2E is a platform for analyzing the behaviour of large and complex software
systems with concolic execution. It introduces its own innovations on top of
refinements made to the original idea. By providing stock \textit{selectors}
and \textit{analyzers} (and the framework for writing your own) for program
execution branches, S2E allows
for \textit{selective symbolic execution} of those paths that are relevant to
the quantity being tested or analyzed. For example, you may only be concerned
with analyzing cache performance, in which case you could write a selector to
explore paths touched by a particular 2D array. S2E will switch to symbolic
execution only when required and will generate concrete values from constraints
in all other cases (for example, when entering the kernel for a syscall). This
increases the effeciency of symbolic execution and allows S2E to analyze
complex real world software stacks. S2E was chosen for this project because of
its performance, programmability, and ability to operate on binaries. 



\bibliographystyle{abbrv} \bibliography{main}

\end{document}
